
Applied Survival Analysis by Hosmer and Lemeshow
%=================================================================================== %
%- https://www.cscu.cornell.edu/news/statnews/stnews78.pdf
%=================================================================================== %
\begin{frame}[fragile]
	
\frametitle{Survival Analysis}

\begin{itemize}
\item Survival analysis is generally defined as a set of methods for analyzing data where the outcome variable
is the time until the occurrence of an event of interest. 
\item The event can be death, occurrence of a disease,
marriage, divorce, etc. 
\item The time to event or survival time can be measured in days, weeks, years, etc.
\item For example, if the event of interest is heart attack, then the survival time can be the time in years until a
person develops a heart attack.
\end{itemize}
\end{frame}
%=================================================================================== %

\begin{frame}[fragile]
\frametitle{Censoring}	


Observations are called censored when the information about their survival time is incomplete; the most
commonly encountered form is right censoring. Suppose patients are followed in a study for 20 weeks.
A patient who does not experience the event of interest for the duration of the study is said to be right
censored. The survival time for this person is considered to be at least as long as the duration of the
study. 
\end{frame}
%=================================================================================== %
\begin{frame}[fragile]
	\frametitle{Censoring}
	\begin{itemize}
		\item
Censoring: dealing with missing data

\item Right censoring:

\item Where the date of death is unknown but is after some known date

\begin{itemize}
\item Date of death is after the end of the study
\item Subject is removed from the study (patient withdraws, animal escapes, plant gets eaten etc.)
\end{itemize}


\item Survival analysis can account for this kind of censoring
\end{itemize}
\end{frame}
%============================================================================================ %
\begin{frame}[fragile]
\frametitle{Censoring}
\begin{itemize}
\item Another example of right censoring is when a person drops out of the study before the end of the
	study observation time and did not experience the event.
	\item  This person’s survival time is said to be
	censored, since we know that the event of interest did not happen while this person was under
	observation. 
	\item Censoring is an important issue in survival analysis, representing a particular type of
	missing data. 
	\item Censoring that is random and non informative is usually required in order to avoid bias in
	a survival analysis. 
\end{itemize}


\end{frame}
%=================================================================================== %

\begin{frame}[fragile]
	
\begin{itemize}
\item The R package(s) needed for this chapter is the survival package. 
\item You may want to make sure that packages on your local machine are up to date. 
\item You can perform updating in R using update.packages() function.
\item Table 1.1 on page 4, data set is hmohiv.csv.
\end{itemize}


\end{frame}
%=================================================================================== %
\begin{frame}[fragile]
 hmohiv<-read.table("http://www.ats.ucla.edu/stat/r/examples/asa/hmohiv.csv", sep=",", header = TRUE)
 attach(hmohiv)
 hmohiv
     ID time age drug censor     entdate     enddate
1     1    5  46    0      1  5/15/1990  10/14/1990 
2     2    6  35    1      0  9/19/1989   3/20/1990 
3     3    8  30    1      1  4/21/1991  12/20/1991 
4     4    3  30    1      1   1/3/1991    4/4/1991 
5     5   22  36    0      1  9/18/1989   7/19/1991 
6     6    1  32    1      0  3/18/1991   4/17/1991 
7     7    7  36    1      1 11/11/1989   6/11/1990 
8     8    9  31    1      1 11/25/1989   8/25/1990 
9     9    3  48    0      1  2/11/1991   5/13/1991 
10   10   12  47    0      1  8/11/1989   8/11/1990 
...........(part of output omitted here)...........
91   91    4  35    0      1  4/10/1990    8/9/1990 
92   92   57  36    0      1 12/11/1990    9/9/1995 
93   93    1  41    1      1 12/15/1990   1/14/1991 
94   94   12  36    1      0  1/13/1989   1/13/1990 
95   95    7  35    1      1  8/22/1991   3/21/1992 
96   96    1  34    1      1   8/2/1991    9/1/1991 
97   97    5  28    0      1  5/22/1991  10/21/1991 
98   98   60  29    0      0   4/2/1990    4/1/1995 
99   99    2  35    1      0   5/1/1991   6/30/1991 
100 100    1  34    1      1  5/11/1989   6/10/1989 
\end{frame}
%=================================================================================== %
\begin{frame}[fragile]
Figure 1.1 on page 6 using the hmohiv data set. To control the type of symbol, a variable called psymbol is created. It takes value 1 and 2, so the symbol type will be 1 and 2.
\begin{framed}
\begin{verbatim}
psymbol<-censor+1
table(psymbol)
psymbol
1  2 
20 80 
plot(age, time, pch=(psymbol))
legend(40, 60, c("Censor=1", "Censor=0"), pch=(psymbol))
\end{verbatim}
\end{framed}

\end{frame}
%=================================================================================== %
\begin{frame}[fragile]
Figure 1.2 on page 7 using the hmohiv data set.

age1<-1000/age
plot(age1, time, pch=(psymbol))
legend(40, 30, c("Censor=1", "Censor=0"), pch=(psymbol))

Table 1.2 on page 14 using the data set hmohiv. Package "survival" is needed for this analysis and for most of the analyses in the book.
\end{frame}
%=================================================================================== %
\begin{frame}[fragile]
library(survival)
test <- survreg( Surv(time, censor) ~ age, dist="exponential")
summary(test)

Call:
survreg(formula = Surv(time, censor) ~ age, dist = "exponential")
             Value Std. Error     z        p
(Intercept)  5.859     0.5853 10.01 1.37e-23
age         -0.094     0.0158 -5.96 2.59e-09

\end{frame}
%=================================================================================== %
\begin{frame}[fragile]
Scale fixed at 1 

Exponential distribution
Loglik(model)= -275   Loglik(intercept only)= -292.3
        Chisq= 34.5 on 1 degrees of freedom, p= 4.3e-09 
Number of Newton-Raphson Iterations: 4 
n= 100 
Figure 1.3 on page 16 using data set hmohiv and the model created for Table 1.2 in previous example.
\end{frame}
%=================================================================================== %
\begin{frame}[fragile]
\begin{framed}
\begin{verbatim}
pred <- predict(test, type="response") 
ord<-order(age)
age_ord<-age[ord]
pred_ord<-pred[ord]
plot(age, time, pch=(psymbol))
lines(age_ord, pred_ord)
legend(40, 60, c("Censor=1", "Censor=0"), pch=(psymbol))
\end{verbatim}
\end{framed}
\end{frame}
%=================================================================================== %
\end{document}