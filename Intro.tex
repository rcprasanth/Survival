1. Introduction

Survival analysis models factors that influence the time to an event. Ordinary least squares regression methods fall short because the time to event is typically not normally distributed, and the model cannot handle censoring, very common in survival data, without modification. Nonparametric methods provide simple and quick looks at the survival experience, and the Cox proportional hazards regression model remains the dominant analysis method. This seminar introduces procedures and outlines the coding needed in SAS to model survival data through both of these methods, as well as many techniques to evaluate and possibly improve the model. Particular emphasis is given to proc lifetest for nonparametric estimation, and proc phreg for Cox regression and model evaluation.

Note: A number of sub-sections are titled Background. These provide some statistical background for survival analysis for the interested reader (and for the author of the seminar!). Provided the reader has some background in surival analysis, these sections are not necessary to understand how to run survival analysis in SAS. These may be either removed or expanded in the future.

Note: The terms event and failure are used interchangeably in this seminar, as are time to event and failure time.
