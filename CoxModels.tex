\subsection*{Cox Proportional Hazard Models}
%========================================
\begin{itemize}
\item Proportional hazards models are a class of survival models in statistics. Survival models relate the time that passes before some event occurs to one or more covariates that may be associated with that quantity of time. \item In a proportional hazards model, the unique effect of a unit increase in a covariate is multiplicative with respect to the hazard rate. \item For example, taking a drug may halve one's hazard rate for a stroke occurring, or, changing the material from which a manufactured component is constructed may double its hazard rate for failure. \item Other types of survival models such as accelerated failure time models do not exhibit proportional hazards. The accelerated failure time model describes a situation where the biological or mechanical life history of an event is accelerated.
\end{itemize}

\begin{framed}
\begin{verbatim}
# Create the simplest test data set 
test1 <- list(time=c(4,3,1,1,2,2,3), 
              status=c(1,1,1,0,1,1,0), 
              x=c(0,2,1,1,1,0,0), 
              sex=c(0,0,0,0,1,1,1)) 

# Fit a stratified model 
coxph(Surv(time, status) ~ x + strata(sex), test1) 
\end{verbatim}
\end{framed}

%============================================================================ %

\begin{framed}
\begin{verbatim}

# Create a simple data set for a time-dependent model 
test2 <- list(start=c(1,2,5,2,1,7,3,4,8,8), 
              stop=c(2,3,6,7,8,9,9,9,14,17), 
              event=c(1,1,1,1,1,1,1,0,0,0), 
              x=c(1,0,0,1,0,1,1,1,0,0)) 
summary(coxph(Surv(start, stop, event) ~ x, test2)) 



\end{verbatim}
\end{framed}


 Create a simple data set for a time-dependent model

%============================================================================ %
\begin{framed}
\begin{verbatim}

test2 <- list(start=c(1, 2, 5, 2, 1, 7, 3, 4, 8, 8),
                stop =c(2, 3, 6, 7, 8, 9, 9, 9,14,17),
                event=c(1, 1, 1, 1, 1, 1, 1, 0, 0, 0),
                x    =c(1, 0, 0, 1, 0, 1, 1, 1, 0, 0) )


summary( coxph( Surv(start, stop, event) ~ x, test2))
\end{verbatim}
\end{framed}
%============================================================================ %
\begin{framed}
	\begin{verbatim}
# Fit a stratified model, clustered on patients 

bladder1 <- bladder[bladder$enum < 5, ] 
coxph(Surv(stop, event) ~ (rx + size + number) * strata(enum) + 
      cluster(id), bladder1)
\end{verbatim}
\end{framed}
%============================================================================ %
\subsection*{ Fit a time transform model using current age}
\begin{framed}
	\begin{verbatim}
	
coxph(Surv(time, status) ~ ph.ecog + tt(age), data=lung,
     tt=function(x,t,...) pspline(x + t/365.25))

\end{verbatim}
\end{framed}
%============================================================================ %
\end{document}