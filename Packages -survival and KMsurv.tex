The package survival is used in this document. Note that the data sets
from Klein and Moeschberger’s book are in the package KMsurv To
obtain one or both of these packages (if they were not previously
installed), use
%%%% ------ 2 / 14
install.packages(’survival’)
install.packages(’KMsurv’)
To load the libraries, use
library(survival)
library(KMsurv)
To view available data sets, use
library(help=KMsurv)
%%%% ------ 3 / 14
Survival object
The survival object is
Surv(time, time2, event,
type=c(’right’, ’left’, ’interval’, ’counting’, ’interval2’, ’mstate’),
origin=0)}
Before any analysis s can be performed, the data has to be put into the
proper format and the function
Surv()
is used to construct these survival objects. Normally the type parameter
can be omitted.
%%%% ------ 4 / 14

Kaplan-Meir
We work with the veteran data set
head(veteran)
fit <- survfit(Surv(time, status) ~ celltype, data=veteran)
plot(fit,col=1:4)
# or
plot(fit,col=1:4,lty=1:4)
legend("topright",legend=c("squamous ","smallcell","adeno","large"),lty=1:4,col=1:4)
%%%% ------ 5 / 14

One can compare the survival or failure times in two or more samples. We
can content ourselves with simple eyeballing and some simple tests. Thus
if survival is unrelated to group effect, then, at each time point, roughly
the same proportion in each group will fail (and conversely will survive).
Statistical tests are based on chi-square-type of statistics that compare the
expected numbers surviving with the observed survival rates.
%%%% ------ 6 / 14
The following five different (mostly nonparametric) tests for censored data
are available:
\begin{itemize}
\item Gehan’s generalized Wilcoxon test
\item Cox-Mantel test
\item Cox’s F test
\item The log-rank test,
\item Peto and Peto’s generalized Wilcoxon test.
\end{itemize}
%%%% ------ 7 / 14

R offers some of these - for example the log-rank test. The log-rank test is
a popular test of the hypothesis that two survival time distributions are
homogeneous. The resulting test statistic is approximately Chi-squared.
%%%% ------ 8 / 14

Logrank test
> summary(coxph(Surv(time, status) ~ celltype, data=veteran))
%%%% ------ 9 / 14

Kidney example
This is a bit more comprehensive We have 119 observations
time - Time to infection, months
delta - Infection indicator (0=no, 1=yes)
type - Catheter placement (1=surgically, 2=percutaneously)
The R analysis gives
%%%% ------ 10 / 14
> 

kidney.surv<-survfit(Surv(time,type)~delta,type="kaplan",data=kidney)
> plot(kidney.surv,col=c(1,2)) # here we use colours 1 and 2

The data is from Klein and Moeschberger (1997) Survival Analysis
Techniques for Censored and truncated data, Springer.
%%%% ------ 11 / 14

To test we use survdiff, the default gives the logrank test
survdiff(Surv(time,type)~delta,data=kidney)
%%%% ------ 12 / 14

Cox’s Proportional Hazard Model
For this he model we assume
\begin{enumerate}
\item The underlying hazard rate (rather than survival time) is a function of
the independent variables (covariates) times a baseline hazard, this is
from equation ??.
\item No assumptions are made about the nature or shape of the hazard
function. Thus, in a sense, Cox’s regression model may be considered
to be a nonparametric method.
\end{enumerate}
%%%% ------ 13 / 14

The table below gives data on patients suffering from leukemia.
time status x time status x
9 1 Maintained 5 1 Nonmaintained
13 1 Maintained 8 1 Nonmaintained
13 0 Maintained 8 1 Nonmaintained
18 1 Maintained 12 1 Nonmaintained
23 1 Maintained 16 0 Nonmaintained
28 0 Maintained 23 1 Nonmaintained
31 1 Maintained 27 1 Nonmaintained
34 1 Maintained 30 1 Nonmaintained
45 0 Maintained 33 1 Nonmaintained
48 1 Maintained 43 1 Nonmaintained
161 0 Maintained 45 1 Nonmaintained
5 1 Nonmaintained - - -
%%%% ------ 14 / 14
The K-M curve
\begin{framed}
\begin{verbatim}
fit=survfit(Surv(time, status)~x, data=aml)
plot(survfit(Surv(time, status)~x, data=aml),col=1:2 )
\end{verbatim}
\end{framed}
We plot or check on the two curves using the log rank test
%%%% ------ 15 / 14
Now we try Cox regression using x as the explanatory variable. In R we
use the command coxph and get the following
\begin{framed}
\begin{verbatim}
> cox1<- coxph( Surv(time, status) ~ x, data = aml)
> summary(cox1)
\end{verbatim}
\end{framed}
