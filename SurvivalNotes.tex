\documentclass[]{beamer}


\begin{document}
kmsurv

Applied Survival Analysis
ats



\section{}
%------------------------------------------------------------------%
\begin{frame}
\frametitle{Survival function}
\Large
The \textbf{survival function}, also known as a survivor function or reliability function, is a property of 
any random variable that maps a set of events, usually associated with mortality or failure of 
some system, onto time. 

\vspace{0.3cm}
It captures the probability that the system will survive 
beyond a specified time.
\end{frame}
%------------------------------------------------------------------%
\begin{frame}
\frametitle{reliability function}\Large
\begin{itemize}
\item The term \textbf{reliability function} is common in engineering while the term survival function 
is used in a broader range of applications, including human mortality. 
\item Another name for the survival function is the complementary cumulative distribution function.
\end{itemize}
\end{frame}
%------------------------------------------------------------------%
\begin{frame}
\frametitle{Definition}
\Large
Let $T$ be a continuous random variable with cumulative distribution function $F(t)$ on the interval $[0,\infty)$. 

Its survival function or reliability function is:

\[R(t) = P(\{T > t\})\]

\[R(t) = \int_t^{\infty} f(u)\,du,.\]

\[R(t) =  1-F(t).\]

\end{frame}
%------------------------------------------------------------------%
\begin{frame}
\frametitle{Properties of the Survival Function}
\Large
\begin{itemize}
\item Every survival function $R(t)$ is monotonically decreasing, i.e. $R(u) \le R(t)$ for all $u > t$.
\item The time, t = 0, represents some origin, typically the beginning of a study or the start of operation of some system. 
\item $R(0)$ is commonly unity but can be less to represent the probability that the system fails immediately upon operation.
\item The survivor function is right-continuous.
\end{itemize}

\end{frame}
%------------------------------------------------------------------%

\end{document}
