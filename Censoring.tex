\documentclass[MasterNotes.tex]{subfiles}
\begin{document}

\section*{Censoring}
\subsection*{Types of Censoring}
\begin{description}
\item[Left censoring] – a data point is below a certain value but it is unknown by how much.
\item[Interval censoring] – a data point is somewhere on an interval between two values.
\item[Right censoring] – a data point is above a certain value but it is unknown by how much.
\end{description}


%------------------------------------------------------------------------ %
\subsection*{Type I and II Censoring}
\begin{description}
\item[Type I] censoring occurs if an experiment has a set number of subjects or items and stops the experiment at a predetermined time, at which point any subjects remaining are right-censored.
\item[Type II] censoring occurs if an experiment has a set number of subjects or items and stops the experiment when a predetermined number are observed to have failed; the remaining subjects are then right-censored.
\end{description}
%------------------------------------------------------------------------- %
\begin{itemize}
\item Random (or non-informative) censoring is when each subject has a censoring time that is statistically independent of their failure time. The observed value is the minimum of the censoring and failure times; subjects whose failure time is greater than their censoring time are right-censored.
\item Interval censoring can occur when observing a value requires follow-ups or inspections. Left and right censoring are special cases of interval censoring, with the beginning of the interval at zero or the end at infinity, respectively.
\item Estimation methods for using left-censored data vary, and not all methods of estimation may be applicable to, or the most reliable, for all data sets.
\end{itemize}

\end{document}