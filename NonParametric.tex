Nonparametric estimation of the hazard function

Standard nonparametric techniques do not typically estimate the hazard function directly. However, we can still get an idea of the hazard rate using a graph of the kernel-smoothed estimate. As the hazard function h(t) is the derivative of the cumulative hazard function H(t), we can roughly estimate the rate of change in H(t) by taking successive differences in H^(t) between adjacent time points, ΔH^(t)=H^(tj)−H^(tj−1). SAS computes differences in the Nelson-Aalen estimate of H(t). We generally expect the hazard rate to change smoothly (if it changes) over time, rather than jump around haphazardly. To accomplish this smoothing, the hazard function estimate at any time interval is a weighted average of differences within a window of time that includes many differences, known as the bandwidth. Widening the bandwidth smooths the function by averaging more differences together. However, widening will also mask changes in the hazard function as local changes in the hazard function are drowned out by the larger number of values that are being averaged together. Below is an example of obtaining a kernel-smoothed estimate of the hazard function across BMI strata with a bandwidth of 200 days:

We request plots of the hazard function with a bandwidth of 200 days with plot=hazard(bw=200)
SAS conveniently allows the creation of strata from a continuous variable, such as bmi, on the fly with the strata statement We specify the left endpoints of each bmi to form 5 bmi categories: 15-18.5, 18.5-25, 25-30, 30-40, and >40.
