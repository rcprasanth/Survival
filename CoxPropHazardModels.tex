
%================================================================================%

\subsection{Cox Model}
%- http://www.medicine.ox.ac.uk/bandolier/painres/download/whatis/cox_model.pdf
A Cox model is a statistical technique for exploring the
relationship between the survival of a patient and several
explanatory variables.
\item  Survival analysis is concerned with studying the time
between entry to a study and a subsequent event (such as
death).
\item  A Cox model provides an estimate of the treatment effect
on survival after adjustment for other explanatory variables.
In addition, it allows us to estimate the hazard (or risk) of death
for an individual, given their prognostic variables.
\item  A Cox model must be fitted using an appropriate computer
program (such as SAS, STATA or SPSS). The final model from a
Cox regression analysis will yield an equation for the hazard
as a function of several explanatory variables.
\item  Interpreting the Cox model involves examining the coefficients
for each explanatory variable. A positive regression
coefficient for an explanatory variable means that the hazard
is higher, and thus the prognosis worse. Conversely, a negative
regression coefficient implies a better prognosis for patients
with higher values of that variable.
\end{itemize}
The Cox model is based on a modelling approach to the analysis of survival data. The
purpose of the model is to simultaneously explore the effects of several variables on
survival.
The Cox model is a well-recognised statistical technique for analysing survival
data. When it is used to analyse the survival of patients in a clinical trial, the model allows us
to isolate the effects of treatment from the effects of other variables. The model can also
be used, a priori, if it is known that there are other variables besides treatment that influence patient survival and these variables cannot be easily controlled in a clinical trial. Using the model may improve the estimate of
treatment effect by narrowing the confidence interval. Survival times now often refer to
the development of a particular symptom or to relapse after remission of a disease, as well as to the time to death.

\subsubsection{Why are survival times
censored?}
A significant feature of survival times is that
the event of interest is very rarely observed in
all subjects. For example, in a study to
compare the survival of patients having
different types of treatment for malignant
melanoma of the skin, although the patients
may be followed up for several years, there
will be some patients who are still alive at the
end of the study. We do not know when these
patients will die, only that they are still alive
at the end of the study; therefore, we do not
know their survival time from the start of
treatment, only that it will be longer than
their time in the study. Such survival times
are termed censored, to indicate that the
period of observation was cut off before the
event of interest occurred.
From a set of observed survival times
(including censored times) in a sample of
individuals, we can estimate the proportion of
the population of such people who would survive a given length of time under the same
circumstances. This method is called the product limit or Kaplan–Meier method.
The method allows a table and a graph to be produced; these are referred to as the life table
and survival curve respectively.
Kaplan–Meier estimate of the survivor function The data on ten patients presented in Table 1
refer to the survival time in years following treatment for malignant melanoma of the
skin.

%============================%

Proportional hazards models are a class of survival models in statistics. 


Survival models relate the time that passes before some event occurs to one or more covariates that may be associated with 
that quantity of time. In a proportional hazards model, the unique effect of a unit increase in a covariate is multiplicative 
with respect to the hazard rate. 

This is in contrast to additive hazard models, wherein a treatment may increase ones hazard rate by a fixed amount independently of other
covariates.

For example, taking a drug may halve one's hazard rate for a stroke occurring, or, changing
the material from which a manufactured component is constructed may double its hazard rate for failure. 

Other types of survival models such as accelerated failure time models do not exhibit proportional hazards. 
The accelerated failure time model describes a situation where 
the biological or mechanical life history of an event is accelerated.

%===============================================%
\subsubsection*{Composition}
Consider survival models to consist of two parts
\begin{enumerate}
\item The Hazard Function
\item The Effects Parameter
\end{enumerate}
